%% Beginning of file 'sample.tex'
%%
%% Modified 2005 December 5
%%
%% This is a sample manuscript marked up using the
%% AASTeX v5.x LaTeX 2e macros.

%% The first piece of markup in an AASTeX v5.x document
%% is the \documentclass command. LaTeX will ignore
%% any data that comes before this command.

%% The command below calls the preprint style
%% which will produce a one-column, single-spaced document.
%% Examples of commands for other substyles follow. Uase
%% whichever is most appropriate for your purposes.
%%
%%\documentclass[12pt,preprint]{aastex}
%% manuscript produces a one-column, double-spaced document:

\documentclass[manuscript]{emulateapj}
%\documentclass[manuscript]{aastex}
\usepackage{natbib}
\usepackage{amsmath}
\usepackage{verbatim}
%\usepackage{ps2pdf}

%% preprint2 produces a double-column, single-spaced document:

%% \documentclass[preprint2]{aastex}

%% Sometimes a paper's abstract is too long to fit on the

%% title page in preprint2 mode. When that is the case,
%% use the longabstract style option. 

%% \documentclass[preprint2,longabstract]{aastex}

%% If you want to create your own macros, you can do so
%% using \newcommand. Your macros should appear before
%% the \begin{document} command.
%%
%% If you are submitting to a journal that translates manuscripts
%% into SGML, you need to follow certain guidelines when preparing
%% your macros. See the AASTeX v5.x Author Guide
%% for information.

\newcommand{\lsim}{\mbox{$_<\atop^{\sim}$}}
\newcommand{\gsim}{\mbox{$_>\atop^{\sim}$}}
\newcommand{\vdag}{(v)^\dagger}
\newcommand{\myemail}{scarlata@ipac.caltech.edu}
\newcommand{\lya}{Ly$\alpha$}
\newcommand{\ha}{H$\alpha$}
\newcommand{\hb}{H$\beta$}
\newcommand{\hi}{H~{\small I}}
\newcommand{\hii}{H~{\small II}}
\newcommand{\oii}{[O~{\small II}]}
\newcommand{\oiii}{[O~{\small III}]}
\newcommand{\heii}{He~{\small II}}
\newcommand{\civ}{C~{\small IV}}
\newcommand{\ciii}{[C~{\small III]}}
\newcommand{\nii}{[N~{\small II]}}
\newcommand{\tbd}{{\bf TBD}}
\newcommand{\ecsa}{erg cm$^{-2}$ s$^{-1}$ \AA$^{-1}$}
\newcommand{\ecs}{erg cm$^{-2}$ s$^{-1}$}
\newcommand{\es}{erg s$^{-1}$}
\newcommand{\esa}{erg s$^{-1}$}
\newcommand{\kms}{km s$^{-1}$}
\newcommand{\lsimeq}{\mbox{$_<\atop^{\sim}$}}
\newcommand{\nII}{[N{\sc ii}]}
\newcommand{\degs}{$^{\circ}$}
\newcommand{\Ipar}{I_{\parallel}}
\newcommand{\Iper}{I_{\perp}}
\newcommand{\p}{$P$}
%\citestyle{aa}

%% You can insert a short comment on the title page using the command below.

%\slugcomment{Draft}

%% If you wish, you may supply running head information, although
%% this information may be modified by the editorial offices.
%% The left head contains a list of authors,
%% usually a maximum of three (otherwise use et al.).  The right
%% head is a modified title of up to roughly 44 characters.
%% Running heads will not print in the manuscript style.

\shortauthors{Beck, M. et al.}

%% This is the end of the preamble.  Indicate the beginning of the
%% paper itself with \begin{document}.
 
\begin{document}
%% LaTeX will automatically break titles if they run longer than
%% one line. However, you may use \\ to force a line break if
%% you desire.

\title{Galaxy Zoo: Humans, Machines, Filters, Oh My!!}

%% Use \author, \affil, and the \and command to format
%% author and affiliation information.
%% Note that \email has replaced the old \authoremail command
%% from AASTeX v4.0. You can use \email to mark an email address
%% anywhere in the paper, not just in the front matter.
%% As in the title, use \\ to force line breaks.

\author{Melanie Beck, Lucy Fortson, Chris Lintott, Claudia Scarlata}

% \altaffiltext{3}{Carnegie Observatories}
%Supernova Ltd., Olde Yard Village #131, Northsound Road, Virgin Gorda, British Virgin Islands}
%ICRANet, Piazzale della Repubblica 10, I-65100 Pescara, Italy
%% Notice that each of these authors has alternate affiliations, which
%% are identified by the \altaffilmark after each name.  Specify alternate
%% affiliation information with \altaffiltext, with one command per each
%% affiliation.


%% Mark off your abstract in the ``abstract'' environment. In the manuscript
%% style, abstract will output a Received/Accepted line after the
%% title and affiliation information. No date will appear since the author
%% does not have this information. The dates will be filled in by the
%% editorial office after submission.

\begin{abstract}

Abstract stuff. 
\end{abstract}


\keywords{stuff and things}

\section{Introduction}
Galaxy evolution. Need for morphology. \\
Need for semi-automation -- big upcoming projects, legacy projects. \\
Attempts at automating galaxy classifications -- pros and cons? \\
Revamp Galaxy Zoo to meet future demands -- scaling. \\
Guess what we did? It's super awesome... 


\section{}

\section{}

\bibliographystyle{apj} 
\bibliography{apj-jour,hmbib}

\end{document}

