
%% using aastex version 6
\documentclass[twocolumn]{aastex6}

% These are the available options:
%   manuscript	: onecolumn, doublespace, 12pt fonts
%   preprint	: onecolumn, single space, 10pt fonts
%   preprint2	: twocolumn, single space, 10pt fonts
%   twocolumn	: a two column article. Probably not needed, but here just in case.
%   onecolumn	: a one column article; default option.
%   twocolappendix: make 2 column appendix
%   onecolappendix: make 1 column appendix is the default. 
%   astrosymb	: Loads Astrosymb font and define \astrocommands. 
%   tighten	: Makes baselineskip slightly smaller
%   times	: uses times font instead of the default
%   linenumbers	: turn on lineno package.
%   trackchanges : required to see the revision mark up and print output
%   numberedappendix: Labels appendix sections A, B, ... This is the default.
%   appendixfloats: Needed. Resets figure and table counters to zero

\usepackage{verbatim}
\usepackage{amsmath}
\usepackage{graphicx}
\usepackage{cleveref}
\usepackage{hyperref}
\usepackage{multirow}
\usepackage[normalem]{ulem}

\usepackage{xcolor}

\newcommand{\vdag}{(v)^\dagger}
\newcommand\aastex{AAS\TeX}
\newcommand\latex{La\TeX}
\newcommand{\rr}[1]{$r_{#1}$}
\newcommand{\M}[1]{$M_{#1}$}
\newcommand{\A}{\textit{A}}
\newcommand{\N}{\textit{N}}
\newcommand{\Pf}{$P_{F,0}$}
\newcommand{\Pn}{$P_{N,0}$}
\newcommand{\p}{$p_0$}
\newcommand{\tf}{$t_F$}
\newcommand{\tn}{$t_N$}
\newcommand{\feat}{`Featured'}
\newcommand{\notfeat}{`Not'}
\newcommand{\raw}{GZ2$_{\text{raw}}$}
\newcommand{\deb}{GZ2$_{\text{debiased}}$}


%%%%%%%%%%%%%%%%%%%%%%%%%%%%%%%%%%%%%%%%%%%%%%%%%%%%%%%%%%%%%%
%% The following commented section outlines numerous optional output that
%% can be displayed in the front matter or as running meta-data.
%%
%% You can insert a short comment on the title page using the command below.
%% \slugcomment{Not to appear in Nonlearned J., 45.}
%%
%% If you wish, you may supply running head information, although
%% this information may be modified by the editorial offices.
%%\shorttitle{\aastex sample article}
%%\shortauthors{Schwarz et al.}
%%
%% You can add a light gray and diagonal water-mark to the first page 
%% with this command:
%% \watermark{text}
%% where "text", e.g. DRAFT, is the text to appear.  If the text is 
%% long you can control the water-mark size with:
%% \setwatermarkfontsize{dimension}
%% where dimension is any recognized LaTeX dimension, e.g. pt, in, etc.
%%
%%%%%%%%%%%%%%%%%%%%%%%%%%%%%%%%%%%%%%%%%%%%%%%%%%%%%%%%%%%%%%%%%%%


\begin{document}


\title{Integrating human and machine intelligence in morphology classification tasks with Galaxy Zoo Express}

%% Use \author, \affil, plus the \and command to format author and affiliation 
%% information.  If done correctly the peer review system will be able to
%% automatically put the author and affiliation information from the manuscript
%% and save the corresponding author the trouble of entering it by hand.
%%
%% The \affil should be used to document primary affiliations and the
%% \altaffil should be used for secondary affiliations, titles, or email.

%% Authors with the same affiliation can be grouped in a single
%% \author and \affil call.
\author{Melanie Beck, Claudia Scarlata, Lucy F. Fortson}
\author{Chris J. Lintott}
\author{Melanie A. Galloway, Kyle W. Willett}
\author{B. D. Simmons}
\author{Karen L. Masters}
\author{Phil Marshall}
\author{Hugh Dickinson}
\author{Darryl Wright}


\affil{Minnesota Institute for Astrophysics, University of Minnesota, Minneapolis, MN 55454}
\affil{Department of Physics, University of Oxford, Oxford, UK}
\affil{Center for Astrophysics and Space Sciences, Department of Physics, University of California–San Diego, San Diego, CA}
\affil{ICG Portsmouth, UK}
\affil{IPAC}

%% Somehow you have to put "Einstein Fellow" as a footnote for Brooke!!!

%% Mark off the abstract in the ``abstract'' environment. 
\begin{abstract}
Quantifying galaxy morphology is a challenging yet scientifically rewarding task. As the scale of data continues to increase with upcoming surveys, traditional classification methods will struggle to handle the load. We present a solution to the scale problem through an integration of visual and machine classifications that preserves the best features of both human and automatic classifications. We demonstrate the effectiveness of such a system through a 
re-analysis of visual galaxy morphology classifications collected during the Galaxy Zoo 2
project and combine these with a machine algorithm that learns on a suite of non-parametric 
morphology indicators widely used for automated morphologies. Our method 
provides an order of magnitude improvement in the rate of galaxy morphology 
classifications while maintaining highly accurate, pure and complete samples, and  
requires a minimal amount of computational cost. 
This result has important implications for the development of galaxy morphology identification tasks in the era of Euclid and other large-scale surveys. 

%We implemented one of the first human-machine combos by running a kick ass
%simulation on previous citizen science data in conjunction with machine algorithms. 
%And guess what? We can obtain at least an ORDER OF MAGNITUDE improvement in the 
%efficiency of classification. So we got that going for us. Which is nice. 

\end{abstract}

%% Keywords should appear after the \end{abstract} command. 
%% See the online documentation for the full list of available subject
%% keywords and the rules for their use.
\keywords{galaxy morphology  --- classification --- machine learning}


\section{Introduction} \label{sec:intro}

Astronomers have made use of galaxy morphologies to understand the dynamical structure of these systems since at least the 1930s \citep[e.g.,][]{Hubble1936}. The division between early-type and late-type systems corresponds, for example, with a wide range of parameters from mass to environment and star formation history \citep[e.g.,][]{Shen2003, Peng2010}, and detailed observation of morphological features such as bars and bulges provide information about the history of their host systems \citep[e.g., review by][]{KK04, Masters2010, Simmons2014}. Modern studies of morphology  divide systems into broad classes \citep[e.g.,][]{Conselice2006, Lintott2008, Kartaltepe2015, Peth2016}, but a wealth of information can be gained from identifying new and often rare classes, such as the green peas \citep{Cardamone2009} and beans \citep{Schirmer2016}. 


%While attempts have been made to use proxies such as color or Sersic index for morphology, there is no simple substitute for classifying by shape. 
While Galaxy Zoo has provided a solution that scales visual classification for current surveys \citep{Willett2013, Willett2016, Simmons2016}, upcoming surveys such as LSST and Euclid will require a different approach, one appropriate for tackling very large data sets. Standard visual morphology methods will be unable to contend with the scale of data, and current automated morphologies provide only statistical morphological structure with large uncertainties \cite[e.g.,][]{Abraham1996, Bershady2000}. 
%(Bamford et al. (red spirals), Schawinski et al (blue ellipticals)  which will come online in the next five years The data promised from these massive surveys will provide a wealth of new information, including millions of never before seen galaxies. In order to understand the structure and evolution of these galaxies, methodologies must be developed to handle very large data sets. Methods must be developed which can efficiently and accurately quantify the likelihood of given morphological features being present.


%%%-------------------------------------------------------
%%% FIGURE:     GZ EXPRESS Schematic
%%%-------------------------------------------------------
\begin{figure*}[ht!]
%\figurenum{1}
\plotone{figures/GZExpress_v4.png}
\caption{Schematic of our hybrid system. Humans provide classifications of galaxy images via a web interface. We simulate this with the Galaxy Zoo 2 classification dataset described in section~\ref{sec: data}. Human classifications are processed with an algorithm described in section~\ref{sec: SWAP}. Subjects that pass a set of thresholds are considered human-retired (fully classified) and provide the training sample for the machine classifier as described in section~\ref{sec: machine}. The trained machine is applied to all subjects not yet retired. Those that pass an analogous set of machine-specific thresholds are considered machine-retried. The rest remain in the system to be seen by either human or machine. This procedure is repeated on a nightly timestep. Our results are reported in section~\ref{sec: results}.  \label{fig: schematic}}
\end{figure*}

One solution is full automation, which can be achieved via multiple approaches. 
One approach is to define a set of features which describe the morphology in an N-dimensional space. Decision boundaries within this space define morphological types for each galaxy. Learning these decision boundaries can be achieved through machine learning techniques such as support vector machines \citep{HuertasCompany2008} or principal component analysis \citep{Scarlata2007}.  Another approach is through deep learning, a machine learning technique that attempts to model high level abstractions. Algorithms like convolutional neural networks have been used with impressive accuracy \citep{Dieleman2015, HuertasCompany2015}. However, a drawback to all automated classification techniques is the need for standardized training data, with more complex algorithms requiring more data. Furthermore, that data must be consistent for each survey: differences in resolution and depth can be inherently learned by the algorithm making their application to disparate surveys challenging.  


In this work we seek a system that preserves the best features of both human and automatic classifications, developing for the first time a system that brings both human and machine intelligence to the task of galaxy morphology to handle the scale and scope of next generation data. We demonstrate the effectiveness of such a system through a 
re-analysis of visual galaxy morphology classifications collected during the Galaxy Zoo 2
project, and combine these classifications with a suite of non-parametric 
morphology indicators widely used for automated morphologies. Our method 
provides an order of magnitude improvement in the rate of galaxy morphology 
classifications while maintaining highly accurate, pure and complete samples.



\section{Galaxy Express Overview}
In Galaxy Express (GX) we combine human and machine to 
increase morphological classification efficiency, both in terms of the 
classification rate and required human effort. 
Figure~\ref{fig: schematic} presents a schematic of GX which includes section 
numbers as a shortcut for the savvy reader. We note that transparent portions
 of the schematic represent areas of future work which we explore in section~\ref{sec: visions}. Any system combining human and machine classifications will have a set of generic 
features: a group of human classifiers, at least one machine classifier, and a 
decision engine which determines how these classifications should be combined.


We draw from the Galaxy Zoo 2 (GZ2) classification database which allows us to 
 create simulations of human classifiers (described in section~\ref{sec: data}).
These classifications are used most effectively when processed with SWAP, 
a Bayesian code described in  section~\ref{sec: SWAP}, first developed for the 
Space Warps gravitational lens discovery project~\citep{Marshall2016}. 
These subjects provide the machine's training sample. 

In section~\ref{sec: machine}, we incorporate the machine classifier. We have 
developed a random forest algorithm that trains on measured morphology
indicators such as Concentration, Asymmetry, Gini coefficient and M$_{20}$. 
After a sufficient number of subjects have been classified by humans, 
 the machine is trained and its performance assessed against a validation sample. 
This procedure is repeated nightly and the machine's performance increases with
size of the training sample, albeit with a performance limit. 
Once the machine reaches an acceptable level of 
performance it is applied to the remaining galaxy sample. 
%Images reliably classified by machine are not further classified by humans.

Even with this simple description, one can see that the classification process 
will progress in three phases.  First, the machine will not yet have reached an 
acceptable level of performance; only humans contribute to subject classification.
Second, the machine's performance will improve; both humans and machine 
will be responsible for classification. Finally, machine performance 
will slow; remaining images will likely need to be classified by humans. 
These results are explored in section~\ref{sec: results}. 
This blueprint allows even modest machine learning 
routines to make significant contributions alongside human classifiers and 
removes the need for ever-increasing performance in machine classification.



%%----------------------------------------------------------------------------------------------------------------------------------------------------
%%   Galaxy Zoo 2 Data Description
%%---------------------------------------------------------------------------------------------------------------------------------------------------
\section{Galaxy Zoo 2 Classification Data} \label{sec: data}

Our simulations utilize original classifications made by volunteers during the GZ2 project. 
These data are described in detail in~\cite{Willett2013}, though we provide a brief overview here.  
The GZ2 subject sample consists of 285,962 galaxies identified as the
 brightest 25\% ($r$-band magnitude $< 17$) residing in the SDSS North Galactic 
Cap region from Data Release 7 and included subjects with both spectroscopic and 
photometric redshifts out to $z < 0.25$\footnote{\url{data.galaxyzoo.org}}.
%Of these, 243,500 have spectroscopic redshifts while 42,462 have only photometric redshifts.  

Subjects were shown as color composite images via a web-based interface wherein 
volunteers answered a series of questions pertaining to the morphology of the subject.
With the exception of the first question, subsequent queries were 
dependent on volunteer responses from the previous task creating a complex decision tree. 
Using GZ2 nomenclature,  a \textit{classification} is the total amount of
information about a subject obtained by completing all tasks in the decision tree. 
%A \textit{task} represents a segment of the tree consisting of a \textit{question} and possible \textit{responses}. 
A subject is \textit{retired} after it has achieved a sufficient amount of classification.
%\footnote{\url{zoo2.galaxyzoo.org}}

For our current analysis, we choose the first task in the tree: 
``Is the galaxy simply smooth and rounded, with no sign of a disk?" to which possible 
responses include ``smooth", ``features or disk", or ``star or artifact". 
This serves two purposes: 1) this is one of only two questions in the GZ2
decision tree that is asked of every subject, thus maximizing the amount of data
we have to work with, and 2) our analysis assumes a binary task and this question is 
simple enough to cast as such. 

By combining the ``star or artifact" vote fraction, $f_{artifact}$, 
with the ``features or disk" vote fraction, $f_{features}$ we obtain a binary response.
Here, a vote fraction is simply the fraction of volunteers who voted for a
particular response.  
We define a label for each GZ2 subject as the majority vote fraction, that is,
if $f_{features}$+$f_{artifact} > f_{smooth}$, the galaxy is labeled~\feat, otherwise
it is labeled~\notfeat. 
We note that only 512 subjects in the GZ2 catalog have a majority $f_{artifact}$, 
contributing less than half a percent contamination.

The GZ2 catalog assigns every subject three types of volunteer vote fractions: 
raw, weighted, and debiased. 
Debiased vote fractions are calculated to correct for redshift bias, a task that 
GZX does not perform. 
The weighted vote fractions account for inconsistent or malicious volunteers, 
a task we perform as well. However, because our mechanism is entirely different 
from GZ2, we derive labels from the raw vote fractions (\raw).
 In total, the data consist of over 16 million classifications from 83,943 individual volunteers. 

%Additionally, we define `true' labels for each GZ2 subject to which we can compare labels assigned by Galaxy Zoo Express (GZX).   
%Specifically, we take the majority vote fraction as the label for that subject. 
%If the majority resided under `star or artifact' or `feature or disk', it was labeled as~\feat; otherwise it was labeled~\notfeat.

%As we discuss in Section~\ref{sec: SWAP}, the algorithm we use requires that every volunteer see a subset of subjects that are expertly identified by a member of the science team. 30,984 volunteers identified one or more of our gold standard sample (see Section~\ref{sec: training sample}), thus we use only those classifications made by one of these ``gold standard'' volunteers. We note that these volunteers represent 36\% of all users yet provide nearly 90\% of the total Galaxy Zoo classification data, reducing the total number of classifications available for our simulated runs to approximately 14 million.
%We thus utilize only those classifications made by one of the 30,894 volunteers that 
%identified one or more of our gold standard sample (see Section~\ref{sec: training sample}). 

%%----------------------------------------------------------------------------------------------------------------------------------------------------
%%   Talk About SWAP 
%%----------------------------------------------------------------------------------------------------------------------------------------------------
\section{Efficiency through clever human-vote processing}\label{sec: SWAP}

Galaxy Zoo 2 had a brute-force subject retirement rule whereby each galaxy 
was required to achieve a threshold of approximately forty independent classifications. 
%was considered retired when a target number of classifications had been reached. 
%As the project required a large number of independent classifications for each subject, the retirement threshold was set rather high, typically at forty  individual volunteer classifications.
Once the project reached completion, inconsistent and unreliable volunteers were
down-weighted as described in~\cite{Willett2013}.
%down-weight inconsistent and unreliable volunteers.  
While this process reduces input from any malicious users or `bots', it doesn't 
make efficient use of those who are exceptionally skilled. 



To intelligently manage subject retirement and increase classification efficiency, 
we adapt an algorithm from the Zooniverse project Space Warps~\citep{Marshall2016}, 
which searched for and discovered several gravitational lens candidates in the 
CFHT Legacy Survey~\citep{More2016}.  
Dubbed SWAP (Space Warps Analysis Pipeline),  this algorithm predicted the 
probability that an image contained a gravitational lens given 
volunteers' classifications and experience after being shown a training
sample consisting of simulated lensing events.  We provide a brief overview here.  

The algorithm assigns each volunteer an \textit{agent} which interprets that volunteer's 
classifications. Each agent assigns a 2$\times$2 confusion matrix to their volunteer which encodes
that volunteer's probability of correctly identifying feature `\A',  given that the subject 
actually exhibits feature \A; and the probability of correctly identifying
the absence of feature \A, given that the subject does not exhibit 
that feature. The agent updates these probabilities by estimating them as 
%~(denoted as \N)
\begin{equation}
P(``X" | X, d) \approx \frac{N_{``X"}}{N_{X}}
\end{equation}
where $N_{``X"}$ is the number of classifications the volunteer labeled as type $X$, 
$N_X$ is the number of subjects the volunteer has seen that were actually of type $X$,
and $d$ represents the history of the volunteer, i.e., all subjects they have seen. 
%The software employs two prescriptions for when the agent updates the volunteer's confusion matrix. In Supervised mode the probabilities are only updated after the volunteer identifies a training subject. In Supervised and Unsupervised mode, the agent updates the probabilities after every subject the volunteer identifies.  


%% -------------------------------------------------------------------------------
%%   FIGURE:  VOLUNTEER PROBABILITIES
%% -------------------------------------------------------------------------------
\begin{figure}[t!]
\includegraphics[width=3.7in]{figures/test_user_probs.png}
\caption{Confusion matrices for 1000 GZ2 volunteers where the size of the circle is proportional to the number of training images that volunteer classified. The histograms on top and right represent the distribution of each component of the confusion matrix for all volunteers.  A quarter of GZ2 volunteers are ``Astute"; they are adept at correctly identifying both~\feat~and~\notfeat~subjects. The peaks at 0.5 in both distributions are an artifact of training: 48\% of volunteers see only one training image; only half of their confusion matrix is updated. \label{fig: volunteer training}}
\end{figure}


Each subject is assigned a prior probability that it exhibits feature \A: $P(A) = p_0$. 
When a volunteer makes a classification, $C$, Bayes' Theorem is used to derive how 
that subject's prior probability should be updated into a posterior using elements
of the agent's confusion matrix. 
As the project progresses, each subject's probability is continually updated,
 nudged higher or lower depending on volunteer input.
Probability thresholds can be set such that subjects crossing a threshold
are highly likely to exhibit the feature of interest or the absence thereof. 
These subjects are then considered retired. 

%\begin{equation}
%P(A|C) = \frac{ P(C|A) P(A) }{P(C|A) P(A) + P(C|N) P(N)}
%\end{equation}

%%%-------------------------------------------------------
%%%  FIGURE:    SWAP FIDUCIAL RUN
%%%-------------------------------------------------------
\begin{figure*}[ht!]
\plotone{figures/GZX_eval_and_retirement_baseline_4paper.png}
\caption{Fiducial SWAP simulation demonstrates a factor of 4-5 increase in the rate of subject retirement as a function of GZ2 project time (left axis, light blue) compared with the original GZ2 project (dark blue). The right axis corresponds to the quality metrics (greys). These are calculated using~\raw~labels developed in section~\ref{sec: data} as ``truth" on the subject sample retired by that day of the simulation. Thus, on the final day, over 225K subjects are retired with accuracy of 95.7\%, completeness of 99\%, and purity of 86.7\%. The decrease in purity as a function of time is partially due to the fact that more difficult to classify subjects are retired later in the simulation. Quality metrics can be influenced by careful selection of input SWAP parameters discussed in detail in Appendix~\ref{sec: tweaking swap}.
\label{fig: fiducial run}}
\end{figure*}

\subsection{Volunteer Training Sample}\label{sec: training sample}

A key feature of the original Space Warps project was the training of 
individual volunteers through the use of simulated images.
These were interspersed with real imaging and were 
predominantly shown at the beginning of a volunteer's association with the project. 
Volunteers were provided feedback in the form of a pop-up comment after
classifying a training image. GZ2 did not train volunteers in such a way which
presents a challenge when applying SWAP to GZ2 classifications. 
We describe how we engineer the GZ2 data to mimic the Space 
Warps system as closely as possible.
% though retroactively training volunteers in real time is obviously not possible. 



%We find that the SWAP algorithm does not perform well without the use of designated training images. Furthermore, SWAP performs optimally when these training images re introduced towards the beginning of a volunteer's contribution to the project. This allows each volunteer's confusion matrix to update sufficiently before intense classification of test images commences. 

We create a gold standard sample by selecting $\sim$3500 SDSS galaxies 
representative of the relative abundance of T-Types, a numerical index of a galaxy's stage along 
the Hubble sequence, at $z\sim0$ by considering galaxies that overlap 
with the~\cite{NairAbraham2010} catalog, a collection of $\sim$14K galaxies 
classified by eye into T-Types. 
%This catalog contains $\sim$14K galaxies classified by eye into various T-Types, a numerical index of a galaxy's stage along the Hubble sequence. Though helpful, this particular classification isn't directly comparable to GZ2 as discussed in~\cite{Willett2013}. Therefore, we took the additional step of
Expert classifications were obtained through the Zooniverse platform\footnote{The Project Builder template facility can be found at \url{http://www.zooniverse.org/labs.}}  
from 15 professional astronomers, including members of the Galaxy Zoo science team. 
 The question posed was identical to the original GZ2 question and at least five 
experts classified each galaxy. 
Votes are aggregated and a simple majority provides an expert label for each subject. 
Our final dataset consists of the GZ2 classifications made 
by those volunteers who classify at least one of these gold standard subjects. 
We thus retain for our simulation 12,686,170 classifications from 30,894 unique volunteers. 
Classifications of gold standard subjects are always processed first. 


%%----------------------------------------------------------------------------------------------------------------------------------------------------
%%   Subsection:    		SWAP only (FIDUCIAL RUN)
%%----------------------------------------------------------------------------------------------------------------------------------------------------
\subsection{Fiducial SWAP simulation}\label{sec: fiducial}

Before we run a simulation, a number of SWAP parameters must be chosen: 
 the initial confusion matrix for each volunteer's agent, the subject 
prior probability, and the retirement thresholds. 
For our fiducial  simulation, we initialize all confusion matrices at (0.5, 0.5), 
and set the subject prior probability, \p~$= 0.5$. 
We set the~\feat~threshold, \tf, i.e., the minimum probability for a subject to be retired as~\feat, to $0.99$. Similarly, we set the~\notfeat~threshold, \tn~$= 0.004$. 
In Appendix~\ref{sec: tweaking swap} 
we show that varying these parameters has only a small affect on the SWAP output. 
To simulate a live project, we run SWAP on a time step of $\Delta t = 1$ day, 
during which SWAP processes all volunteer classifications with timestamps 
within that range. This is performed for three months worth of GZ2 classification data. 

Figure~\ref{fig: volunteer training} (adapted from~\cite{Marshall2016})
demonstrates the volunteer assessment we achieve, and shows confusion matrices 
 for 1000 randomly selected volunteers. The circle size is proportional to the number of 
gold standard subjects that volunteer classified. 
The histograms represent the distribution of each component
of the confusion matrix for all volunteers.
 Nearly 25\% of volunteers are considered `Astute'  indicating
they are generally good at correctly identifying both~\feat~and~\notfeat~subjects.
The spikes at $0.5$ in the histograms are due to volunteers who see only one 
gold standard subject (i.e.,~\feat), leaving their probability in the 
other (\notfeat) unchanged.
Additionally, 4\% of volunteers have a confusion matrix of (0.5, 0.5) indicating these 
volunteers classified two gold standard subjects of the same type, one correctly and 
one incorrectly. 
%the 48\% of
%However, before a simulation can be run, a number of parameters which control the behavior of SWAP must first be chosen. These include the initial confusion matrix assigned to each volunteer, the retirement thresholds, and the prior probability of the subject. Specifically, we must choose 
%\begin{itemize} 
%\item \Pf, the initial probability that a volunteer identifies a subject as being~\feat, $P_0(``F"|F)$
%\item \Pn, the initial probability that a volunteer identifies a subject as being~\notfeat, $P_0(``N"|N)$
%\item \p, the prior probability of a subject to be~\feat.
%\item \tf, the threshold defining the minimum probability for a subject to be retired as~\feat.
%\item \tn, the threshold defining the maximum probability for a subject to be retired as~\notfeat.
%\end{itemize}

%%%-------------------------------------------------------
%%%  FIGURE:   SWAP vote distributions
%%%-------------------------------------------------------
\begin{figure}[t!] 
\includegraphics[width=3.65in]{figures/GZX_clicks_till_retired_baseline.png}
\caption{SWAP requires 4-5 times less human effort than GZ2 as evidenced by the distribution of the number of classifications a subject requires for retirement for the $\sim$225K subjects retired during our fiducial run.  As expected, the GZ2 distribution peaks around 45 classifications per subject. In contrast, most subjects need only 9 classifications when processing with SWAP.  Furthermore,  `easy' subjects can reach retirement in as few as 3-4 classifications. \label{fig: swap vote distributions}}
\end{figure}

Our goal is to increase the efficiency of galaxy classification. We therefore
 use as a metric the cumulative number of retired subjects
as a function of the original GZ2 project time.
%GZ2 retirement was defined by the number of volunteer classifications.  
%requiring $\sim$40 individual volunteers to reach classification consensus for each subject. 
We define a subject as GZ2-retired once it achieves 30 volunteer votes since 
not every subject reached the 40 volunteer threshold~\citep{Willett2013}. 
A subject is considered SWAP-retired once its posterior 
probability crosses either of the retirement thresholds defined above. 

However, it is important not to prioritize efficiency at the expense of quality. 
We thus also consider the metrics of accuracy, 
purity and completeness as a function of GZ2 project time.  These are 
defined as follows: accuracy is the number of correctly
identified subjects divided by the total number retired; completeness is the number of 
correctly identified~\feat~subjects divided by the number of actual~\feat~retired; 
and purity is the number of correctly identified~\feat~subjects divided by 
the number of subjects retired as~\feat. Thus, a complete sample has no false
negatives whereas a pure sample has no false positives. 
We compute these metrics by comparing the SWAP-assigned labels of the cumulatively 
retired subject set to the~\raw~labels for each day of the simulation. 
For example, by Day 20, SWAP retires 120K subjects with 96\% accuracy,
 99.7\% completeness, and 92\% purity. 
%Furthermore, the metrics computed on the last day of the simulation reflect the quality of the full SWAP-retired sample. 
%(that is, of all the subjects we retire up to that point, we successfully identify all that are~\feat),and 92\% pure. 

Figure \ref{fig: fiducial run} and Table~\ref{tab: summary} detail the results of 
our fiducial SWAP simulation compared to the original GZ2 project. 
The left axis shows the cumulative number of retired subjects as a function of 
GZ2 project time. By the end of our simulation, GZ2 retires $\sim$50K subjects while 
SWAP retires 226,124 subjects.  
We thus classify 80\% of the entire GZ2 sample in three months. 
The original GZ2 project took approximately one year to classify as many subjects, 
representing a factor of four increase in the classification rate.  
The right axis of Figure~\ref{fig: fiducial run} corresponds to the quality of 
those classifications as a function of time and establishes that our full 
SWAP-retired sample is 95.7\% accurate, 99\% complete, and 86.7\% pure.

There is also a reduction in the human effort required to perform this classification task.
  Figure~\ref{fig: swap vote distributions} shows the distribution of the number 
of volunteer classifications per subject achieved through SWAP (light blue) 
and GZ2 (dark blue) for the 226K subjects retired in our fiducial run. 
GZ2's distribution peaks at $\sim$45 indicating that, on average,
45 unique volunteers classify each subject. On the other hand, SWAP's
distribution peaks around 9 classifications per subject. 
Furthermore, subjects that are `easy' to classify (i.e.,~\feat) require
even fewer classifications to reach strong consensus. 
More precisely, SWAP processes $2.3 \times 10^6$ volunteer 
classifications while GZ2 records $\sim$$10^7$ for the same subject set. 
SWAP reduces human effort by more than a factor of four. 
%Some subjects are `easy' to classify and can be retired in as few as three classifications, while subjects with less consensus will take more classifications, each one kicking the subject back and forth in probability space before it eventually crosses one of the retirement thresholds. This explains the tail towards larger classification number.


As we demonstrate in Appendix~\ref{sec: tweaking swap}, varying the initial SWAP 
parameters from the fiducial values does not substantially change the results 
presented here. The largest influence comes from choosing unrealistic subject 
prior probabilities which can mildly degrade the quality of the resulting classifications. 
More importantly, none of these effects significantly alters our human and machine integration. 

%We have shown that by clever and adaptive processing of volunteer classifications, efficiency of subject retirement can be increased by a factor of four. The rate and quality of subject retirement will be, in part, a function of initial SWAP parameters. We explore in depth the SWAP parameter space in Appendix~\ref{sec: tweaking swap}.
%We now turn to a discussion of how classification and retirement change as a function of SWAP input parameters. 


\subsection{Disagreements between SWAP and GZ2}

%%%-------------------------------------------------------
%%%  FIGURE:   SWAP failures
%%%-------------------------------------------------------
\begin{figure}[t!]
\includegraphics[width=3.2in]{figures/swapgetsitwrong_cropped.png}
\caption{Distribution of GZ2 $f_{features} + f_{artifact}$ vote fractions for subjects identified as false positives (blue), false negatives (red) and the full sample retired by SWAP (solid black line). The proportions are independent for each of the three distributions. From section~\ref{sec: data}, subjects with values $> 0.5$ are defined as~\feat, however, the red distribution indicates that SWAP labels them as~\notfeat. This is not a flaw of SWAP: 68.9\% of incorrectly identified subjects have $0.4 \le f_{features} + f_{artifact} \le 0.6$ suggesting that~\raw~labels are simply too uncertain. \label{fig: SWAP sucks}}
\end{figure}
%We note that the overlap of the false positive distribution onto the false negatives represents those subjects that have vote fractions exactly at 50-50.

Galaxy Zoo's strength comes from the consensus of dozens of volunteers voting on each subject. 
Processing votes with SWAP reduces the number of classifications to reach consensus. 
Though we typically recover the~\raw~label, SWAP disagrees about 5\% of the time. 
We thus examine the false positives (subjects SWAP labels as~\feat~but~\raw~labels as~\notfeat) and false negatives (subjects SWAP labels as~\notfeat~but~\raw~labels as~\feat).


%In this section we examine the effects driving this disagreement. 
%whereby we again  turn to our fiducial simulation, 
We find the majority of these disagreements are due to uncertainties in 
the~\raw~label. Figure~\ref{fig: SWAP sucks} shows the distribution of $f_{features} + f_{artifact}$ for the false positives (blue), and the false negatives (red).  
Recall that if this value is greater than 0.5, the subject is labeled~\feat.  
The majority of incorrectly 
labeled subjects have $0.4 \le f_{features}+f_{artifact} \le 0.6$, indicating that 
the GZ2 raw vote fractions are simply too uncertain to provide high quality labels. 
We note that the distribution overlap is due to subjects that do not have a majority;
these are labeled~\feat~by default. 

Two other effects contribute to the disagreement between SWAP and GZ2. 
First, as the number of classifications used to retire a galaxy decreases, the 
likelihood of misclassification by random chance increases. 
Second, disagreement arises due to expert-level volunteers whose confusion 
matrices are close to 1.0. These volunteers are essentially more 
strongly weighted, allowing that subject's posterior to cross a retirement threshold
in as few as two classifications. In some cases, these expert-level volunteers disagreed 
with the majority. These issues can be mitigated by requiring each subject reach 
a minimum number of classifications before allowing its probability to cross a 
threshold, thus combining the best qualities of GZ2 and SWAP. 


\subsection{Summary}

We demonstrate a factor of four or more increase in 
classification efficiency while maintaining 95\% accuracy, nearly perfect 
completeness of~\feat~subjects, and with a purity that can be controlled by careful 
selection of input parameters to be better than 90\% (see Appendix~\ref{sec: tweaking swap}).
Exploring those subjects wherein SWAP and GZ2 disagree, we conclude that 
the majority of this disagreement stems from uncertainty in~\raw~labels.
We now turn our focus towards incorporating a machine
classifier utilizing these SWAP-retired subjects as a training sample. 


%%----------------------------------------------------------------------------------------------------------------------------------------------------
%%   INTEGRATING MACHINE CLASSIFIERS 
%%----------------------------------------------------------------------------------------------------------------------------------------------------
\section{Efficiency through incorporation of machine classifiers} \label{sec: machine}

We construct the full Galaxy Express by incorporating supervised 
learning, the machine learning task of inference from labeled training data. 
The training data consist of a set of training examples, and must include
an input feature vector and a desired output label.  Generally speaking,
a supervised learning algorithm analyzes the training data and produces a 
function that can be mapped to new examples. An optimized algorithm will 
correctly determine class labels for unseen data. By processing human classifications 
through SWAP, we obtain a set of binary labels by which we can train a machine 
classifier. We briefly outline the technical details of our machine below,  turning
towards the decision engine we develop in section~\ref{sec: decision engine}. 
%In general, most classification algorithms can handle prediction of several labels simultaneously. 
%Work has been done to predict the entirety of GZ2 classification labels using deep learning algorithms~\citep{Dieleman2015} with great success.  However, it is still simpler for a machine to predict fewer labels than it is to predict several dozen, \textbf{[citation?]}, with the additional bonus that fewer class labels require less training data. 


%%----------------------------------------------------------------------------------------------------------------------------------------------------
%%   RANDOM FORESTS
%%----------------------------------------------------------------------------------------------------------------------------------------------------
\subsection{Random Forests}
%Because our task is simple, we choose a simple machine. In particular, 
We use a Random Forest (RF) algorithm~\citep{Breiman2001},  
an ensemble classifier that operates by
 bootstrapping the training data and constructing a multitude of individual decision 
tree algorithms, one for each subsample.  
An individual decision tree works by deciding which of 
the input features best separates the classes. It does this by performing 
splits on the values of the input feature that minimize the classification 
error. These feature splits proceed recursively. Decision trees alone are
 prone to overfitting, precluding them from generalizing 
well to new data. Random Forests mitigate this effect by combining the 
output label from a multitude of decision trees.  In particular we use the 
\texttt{RandomForestClassifier} from the Python module \texttt{scikit-learn}
\citep{scikit-learn}. 


%%----------------------------------------------------------------------------------------------------------------------------------------------------
%%   CROSS-VALIDATION
%%----------------------------------------------------------------------------------------------------------------------------------------------------
\subsection{Cross-validation}
Of fundamental importance is the task of choosing an algorithm's hyperparameters, 
values which determine how the machine learns.  In the case of a RF, these include
 the maximum depth of the tree, the minimum leaf size, the maximum
number of leaf nodes, among others. The goal is to determine which values will optimize 
the machine's performance and thus these values cannot be chosen \textit{a priori}. 
We perform a $k$-fold cross-validation whereby the 
training sample is split into $k$ subsamples. One subsample is withheld to 
estimate the machine's performance while the remaining data is used to train the machine. 
This is performed $k$ times and the average performance
value is recorded. The entire process is repeated for every combination of the 
 hyperparameter space and values that optimize the output are chosen. 

%Ideally, one would train the machine with every combination of parameters and consider the resulting performance by testing the trained machine on a sample withheld from the training sample so as not to contaminate the results. Formally, 
 

%%----------------------------------------------------------------------------------------------------------------------------------------------------
%%   FEATURE REPRESENTATION AND PRE-PROCESSING
%%----------------------------------------------------------------------------------------------------------------------------------------------------
\subsection{Feature Representation and Pre-Processing}
The feature vector on which the machine learns is composed of D individual 
numeric quantities associated with the subject that the machine uses to discern 
that subject from others in the training sample. 
To segregate~\feat~from~\notfeat, we draw on ZEST \citep{Scarlata2007} and compute
 concentration, asymmetry, Gini coefficient, and M$_{20}$, 
the second-order moment of light for the brightest 20\% of galaxy pixels 
(Appendix \ref{sec: measuring morphology} details the measurement process). 
Coupled with SExtractor's measurement of ellipticity \citep{sextractor}, 
we provide the machine with a five dimensional morphology parameter space. 
These non-parametric diagnostics have long been used to 
quantify galaxy morphology in an automated fashion \cite[e.g.,][]{Abraham1996, Bershady2000, Conselice2000, Abraham2003, Conselice2003, Lotz2004, Snyder2015}.
Because the RF algorithm handles a variety of input formats, the only preprocessing step we perform is the removal of poorly-measured morphological indicators, i.e. catastrophic failures.



%%%-------------------------------------------------------
%%%  FIGURE:   LEARNING CURVE
%%%-------------------------------------------------------
\begin{figure}[t!]
\includegraphics[width=3.25in]{figures/learning_curve_RF.png}
\caption{Learning curve for a Random Forest with fixed hyperparameters. The training score is the accuracy of the trained machine applied to its own training sample. The cross-validation score is the accuracy of the machine computed during the cross-validation process. When the training sample size is small, the machine accurately identifies its own training sample but is unable to generalize to unseen data, as evidenced by a low cross-validation score. As the training sample size increases, the cross-validation score increases. This behavior plateaus indicating that larger training sample sizes provide little in additional performance. \label{fig: learning curve}}
\end{figure}


%%----------------------------------------------------------------------------------------------------------------------------------------------------
%%   DECISION ENGINE 
%%----------------------------------------------------------------------------------------------------------------------------------------------------
\subsection{Decision Engine}\label{sec: decision engine}
A number of decisions must be addressed before attempting to train the machine. 
In particular, which subjects should be designated as the training sample? 
When should the machine attempt its first training session? 
When has the machine's performance been optimized such that it will successfully
generalize to unseen subjects? The field of machine learning provides few hard rules 
for answering these questions, only guidelines and best practices. 
Here we briefly discuss our approach for the development of our decision engine.
%How do we decide when the machine has successfully trained enough to be applied to unseen subjects? 

As discussed in detail in section~\ref{sec: SWAP}, SWAP yields a probability that 
a subject exhibits the feature of interest. While some machine algorithms can 
accept continuous input labels, the RF requires distinct classes. We thus use only 
those subjects which have crossed either of the retirement thresholds. 
Though we find that SWAP consistently retires 35-40\%~\feat~subjects on 
any given day of the simulaton, a balanced ratio of~\feat~to~\notfeat~isn't guaranteed.
 Highly unbalanced training samples should be resampled to correct the imbalance; 
however, as we exhibit only a mild lopsidedness, we allow the machine to train on all 
SWAP-retired subjects.  

SWAP retires a few hundred subjects during the first days of the simulation.
In principle,  a machine can be trained with such a small sample, but will be unable
to generalize to unseen data. We estimate a minimum number of training samples
and the machine's ability to generalize by considering a learning curve, an illustration
of a machine's performance with increasing sample size for fixed hyperparameters. 
Figure~\ref{fig: learning curve} demonstrates such a curve wherein we plot
the accuracy from both the $k$-fold cross-validation, and the trained machine
applied to its own training sample.  
When the sample size is small, the cross-validation score is low and the training 
score is high, a clear sign of over-fitting.  However, as the training 
sample size increases, the cross-validation score increases and eventually plateaus,
 indicating that larger training sets will yield little additional gain. 

We estimate this plateau begins when the training 
sample reaches 10,000 subjects and require SWAP retire at least this many 
 before the machine attempts its first training.  We estimate the machine 
has trained sufficiently if the cross-validation score fluctuates by less than 1\% 
for three consecutive nights of training to ensure we have reached the plateau.  
This requires that we record the machine's training performance each night, 
including how well it scores on the training sample, the 
cross-validation score, and the best hyperparameters. 

%%%-------------------------------------------------------
%%%  FIGURE:    FULL GZX PERFORMANCE
%%%-------------------------------------------------------
\begin{figure}[t!]
\raggedleft
\includegraphics[width=3.45in]{figures/GZ2_sup_PLPD5_p5_flipfeature2b_RF_accuracy_redo_raw_combo_moneyplot.png}
\caption{Galaxy Express reduces the classification rate by an order of magnitude after incorporating a machine classifier, retiring more than 200K subjects in just 27 days. The dashed line marks the first night the machine trains. After several additional nights of training, it is deemed optimized and allowed to retire subjects. Both humans and machine then contribute to retirement. We end the simulation after 32 days having retired over 210K galaxies. See Table~\ref{tab: summary} for details. \label{fig: money}}
\end{figure}

%%----------------------------------------------------------------------------------------------------------------------------------------------------
%%   MACHINE SHOP 
%%----------------------------------------------------------------------------------------------------------------------------------------------------
\subsection{The Machine Shop}\label{sec: machine shop}
We can now describe a full GZX simulation, which begins with human classifications 
processed through SWAP for several days.   
Once at least 10K subjects have been retired, their feature vectors are passed to 
the machine for its inaugural training. 
A suite of performance metrics are recorded by a machine agent, similar
in construction to SWAP's agents. This agent determines 
when the machine has trained sufficiently by assessing the variation
in performance metrics for all previous nights of training. 
Once the machine has been optimized, the agent introduces it to the test sample
consisting of any subject that has not yet reached retirement through SWAP 
and is not part of the gold standard sample.  

Analogous to SWAP, we generate a retirement rule for machine-classified subjects. 
In addition to the class prediction, the RF algorithm computes the probability for
each subject to belong to each class.  This probability is simply the average of
 the probabilities of each individual decision tree, where the probability of a 
single tree is determined as the fraction of subjects of class X on a leaf node.  
Only subjects that receive a class prediction with $p_{machine} \ge 0.9$ are considered
retired. The remaining subjects have the possibility of being classified by humans 
or the machine on a future night of the simulation. 
This constitutes the core of our passive feedback mechanism. Subjects that are
not retired by the machine can instead be retired by humans, thus providing 
the machine a more fully sampled morphology parameter space on future 
training sessions. 



%%----------------------------------------------------------------------------------------------------------------------------------------------
%%   RESULTS 
%%----------------------------------------------------------------------------------------------------------------------------------------------
\section{Results} \label{sec: results}
We perform a full GZX simulation incorporating our RF with the fiducial 
SWAP run discussed in section~\ref{sec: fiducial}. 
The machine attempts its first training on Day 8 with an initial training
sample of $\sim$20K subjects. It undergoes several additional nights 
of training, each time with a larger training sample. 
By Day 12, SWAP has provided over 40K subjects for training and the machine's 
agent has deemed the machine optimized. 
The machine predicts class labels for the remaining 230K GZ2 subjects. 
Of those, the machine retires over 70K, dramatically increasing the 
subset of retired subjects. 
We end the simulation after 32 days, having retired $\sim$210K subjects
as detailed in Table~\ref{tab: summary}. 

We present these results in Figure~\ref{fig: money} where subject retirement 
with GZX (red) is compared to our fiducial SWAP-only run (light blue) and GZ2 (dark blue). 
%The top panel of Figure~\ref{fig: money} shows our usual quality metrics for both SWAP-only and GZX, again using the~\raw~labels discussed in section~\ref{sec: data}.
Using the~\raw~labels as before, we compute our usual quality metrics on the 
full sample of GZX-retired subjects. Accuracy and purity remain within a few
percent of the SWAP-only run at 93.5\% and 84.2\% respectively. 
Instead we see a 5\% decline in the completeness. 
While the SWAP-only run identified 99\% of~\feat~subjects, incorporation
of the machine seems to miss a significant portion thus dropping GZX completeness to 94.3\%. 
We discuss this behavior below.

By dynamically generating a training sample through a more sophisticated analysis of 
human classifications coupled with a machine classifier, we retire more than 200K 
GZ2 subjects in just 27 days.  Visual classification through SWAP alone retires as 
many in 50 days, while GZ2 requires a full year.  
GZX thus provides an order of magnitude increase in the rate of classification
over the traditional crowd-sourced approach. 
We next explore the composition of those classifications.

\begin{table*}[]
	\centering
	\caption{Summary of key quantities for GZ2 and our various simulations. All quality metrics are calculated using~\raw~labels.}
	\label{tab: summary}
	\let\mc\multicolumn
	\begin{tabular}{lcccccc}
		
		\mc7c{ \textbf{Simulation Summary} } \\
		\hline \hline
			& Days	& Subjects Retired & Human Effort 	&  Accuracy 	& Purity 	& Completeness\\
		\mc2c{} 		& 	 	& (classifications) 	&  (\%)	    	& (\%)	& (\%)	\\
		\hline
			
		Galaxy Zoo 2	&	430 	& 285962  	& 16,340,298 	& --   	& --    	 & --   \\
		SWAP only	&	92    	& 226124          & 2,298,772	& 95.7 	& 86.7	 & 99.0     \\
		SWAP+RF      	& 32  	& 210543 	& 932,017 	& 93.5    	& 84.2    	& 94.3      \\
		\hline
	\end{tabular}
\end{table*}

%%----------------------------------------------------------------------------------------------------------------------------------------------
%%   WHO RETIRES WHAT, WHEN? 
%%----------------------------------------------------------------------------------------------------------------------------------------------
\subsection{Who retires what, when?}  

In the top panel of Figure~\ref{fig: gzx components} we explore the individual 
contributions to GZX subject retirement from the RF (blue) and SWAP (red). 
The solid black line shows the total GZX retirement (SWAP+RF), while the dashed line depicts 
the fiducial SWAP-only run from section~\ref{sec: fiducial} for reference. 
Two things are immediately obvious. First, each component shoulders approximately
half of the retirement burden with the machine and SWAP responsible for $\sim$$100$K and $\sim$$110$K subjects respectively.  
	Secondly, the rate of retirement exhibited by the two components is in stark contrast.
SWAP retires at a relatively constant rate while the machine retires 
dramatically at the beginning of its application, quickly surpassing the human 
contribution, and plateaus thereafter. 


%%%-------------------------------------------------------
%%%  FIGURE:    GZX COMPONENT CONTRIBUTIONS
%%%-------------------------------------------------------
\begin{figure}[t!]
\includegraphics[width=3.3in]{figures/GZ2_sup_PLPD5_p5_flipfeature2b_RF_accuracy_redo_raw_combo_GZX_component_contributions.png}
\caption{The top panel demonstrates that the relative contributions to subject retirement are nearly equal between the human (red) and machine (blue) components but display different behaviors over the course of the simulation with the dashed line showing the fiducial SWAP-only run for comparison. The bottom panels shows what fraction of GZ2 subjects are retired, separated by class label. Overall, GX retires 73.6\% of the entire GZ2 sample in 32 days distributed evenly between~\feat~and~\notfeat~as indicated by the black lines. However, humans retire 30\% more~\feat~subjects than the machine, while both components retire a similar proportion of~\notfeat~subjects. \label{fig: gzx components}}
\end{figure}


We thus clearly see three epochs of subject retirement, as we presumed.
In the first phase, humans are the only contributors to subject retirement.  
Once the machine is optimized, it immediately contributes more to retirement than humans.
However, the machine's performance plateaus quickly;  the third 
phase is again dominated by human classifications.

In the bottom panels of Figure~\ref{fig: gzx components}, we consider the class
composition of subjects retired by SWAP and the RF. 
The left (right) panel shows the retired fraction of GZ2 subjects identified 
as~\feat~(\notfeat) according to their~\raw~labels as a function of GZ2 project time. 
Overall, GZX retires 73.6\% of the GZ2 subject sample and this is evenly 
distributed between~\feat~and~\notfeat~subjects as indicated by the solid
black lines in both panels. 
However, SWAP retires more than 50\% of all~\feat~subjects while the machine
retires only 18\%. This divergence does not exist for~\notfeat~subjects where
each component contributes 33-37\%. 

What is the source of this discrepancy? 
Each night the machine trains on a sample composed consistently of 30-40\%~\feat~subjects but does not retire a similar proportion, indicating
that the 30\% of non-retired~\feat~subjects do not receive high $p_{machine}$. 
%Are humans simply better at identifying featured galaxies with the machine optimized to identify the~\notfeat~subjects? 
In the following section we explore whether this is an artifact of our choice in machine 
or in the human-machine combination implemented here. 
% here, or something else entirely? In the next section we explore the machine's performance in the context of its training through human classifiers. 


\subsection{Machine performance}\label{sec: machine performance}

%%%-------------------------------------------------------
%%%  FIGURE:    MACHINE GETS IT RIGHT!!!
%%%-------------------------------------------------------
\begin{figure*}[t!]
\centering
\includegraphics[width=6in]{figures/machine_false_positives_jpgs.png}
\caption{A subsample of galaxies labeled~\notfeat~according to~\raw~labels but which the machine identifies as~\feat. We display $f_{smooth}$ in the lower left corner, that is, the fraction of volunteers who classified the subject as `smooth' (\notfeat). Values are typically between 0.5 and 0.65 indicating that GZ2 does not reach a strong consensus. Fortunately, the machine is able to identify these subjects as~\feat~due to their measured morphology diagnostics. \label{fig: machine false pos}}
\end{figure*}


Throughout our analysis we have defined~\feat~and~\notfeat~subjects by 
their~\raw~labels as this was the most compatible choice for comparison with SWAP output.  
However, the machine does not learn in the same way, nor is it presented with the 
same information. We argue that the machine classifications are, in fact, valid 
and complimentary to human classifications. 

We visually examine several hundred subjects that were deemed false positives, 
i.e., galaxies retired as~\feat~by the machine which have~\notfeat~\raw~labels. 
Figure~\ref{fig: machine false pos} shows a random sample of images we inspected. 
The value in the lower left corner is the raw GZ2 smooth vote fraction, $f_{smooth}$; 
the fraction of volunteers who classified that subject as~\notfeat. The sample
consists primarily of edge-on disks and disk galaxies with low surface brightness
features. 

That the machine can identify~\feat~galaxies that humans classify 
as~\notfeat~has two contributing factors: 
1) the first task of the GZ2 decision tree asks a very specific question that 
does not necessarily correlate with a split between early- and late-type galaxies, and 
 2) the machine learns on morphology diagnostics that are very different from visual inspection. 
% is a recognition that many humans will interpret a task quite literally, 
Regarding the first point, we see that $f_{smooth}$ is generally
 between 0.5 and 0.65 for this sample, indicating that volunteers have not 
reached a strong consensus. This behavior could be modified by providing
actual training images and live feedback as performed in \cite{Marshall2016}. 
The second point suggests that the morphology indicators we measure are 
sufficient for the machine to recognize~\feat~galaxies regardless of the labels humans provide. 
It remains to be seen whether this generalizes to other machine algorithms.
We note that most of these galaxies are labeled~\feat~after GZ2's debiasing process. 

%In our system, the machine classifies subjects which are not seen by humans. However, the disagreement between human and machine is an area ripe for discovery. 
The complementary nature of human and machine classification can 
best be utilized by a feedback mechanism in which a portion of machine-retired
subjects are reviewed by humans. Subjects that display excessive disagreement
should be verified by an expert (or expert-user).  In the same way that 
humans increase the machine's training sample over time, subjects that the
machine properly identifies can become part of the humans' training sample. 
 
%%%-------------------------------------------------------
%%%  FIGURE:   CAS DISTRIBUTIONS 
%%%-------------------------------------------------------
%\begin{figure*}[t!]
%\centering
%\includegraphics[width=7in]{figures/GZ2_sup_PLPD5_p5_flipfeature2b_RF_accuracy_redo_raw_combo_morph_params_raw_labels_4paper.png}
%\caption{What the machine can and can't classify. \label{fig: morph params}}
%\end{figure*}


%%----------------------------------------------------------------------------------------------------------------------------------------------------
%%   MAD RAVINGS OF THE FUTURE 
%%----------------------------------------------------------------------------------------------------------------------------------------------------
\section{Looking Forward}\label{sec: visions}

We have demonstrated the first practical framework for combining human and machine
 intelligence in galaxy morphology classification tasks. 
%By reprocessing the original Galaxy Zoo 2 classifications with SWAP, incorporating a supervised machine learningalgorithm, and implementing a simple decision engine which guides how these two agents interact, we achieve an order of magnitude increase in the classification rate. 
%This is achieved while maintaining accurate identification of both~\feat~and~\notfeat~
%galaxies to over 95\%. Additionally, we recover nearly 95\% of~\feat~galaxies with contamination at a modest 15\%. We now discuss our road map for the future including implications for upcoming massive surveys such as LSST or Euclid. 
While we focus below on a brief discussion of our next steps and potential applications
to large upcoming surveys, we note that our results have implications for the future
of citizen science and Galaxy Zoo in particular. 


GX is perhaps one of the simplest ways to combine human and machine intelligence
 and its impressive performance motivates a higher level of sophistication.
Towards this end, we envision multiple forms of active feedback in addition to 
our passive feedback mechanism.  SWAP allows us to leverage the 
most skilled volunteers to review galaxies difficult for either
 human or machine to classify.  Additionally, machine-retired subjects should 
contribute to the training sample for humans in an analogous fashion to what 
we have already implemented.
%a portion of subjects classified by machine should subsequently be reviewed by humans since we have shown that disagreements between these two agents could signify to experts that something interesting is afoot. 
%This could additionally provide a timely warning: if the machine consistently fails the spot-checking, a new algorithm could instead take its place. as a means of spot-checking that the machine is performing as expected as well 

Secondly, the random forest algorithm is not easily adapted to handle measurement 
errors or class labels with continuous distributions. To fully utilize the information 
provided by SWAP, sophisticated algorithms such as deep convolutional neural 
networks (CNN) or Latent Dirichlet allocation (LDA), an algorithm that is frequently 
used in document processing, should be considered.  Furthermore, there is no reason 
to limit to a single machine. As hinted at in Figure~\ref{fig: schematic}, several 
machines could train simultaneously, their predictions aggregated through 
SWAP, creating an on-the-fly machine ensemble.

%%----------------------------------------------------------------------------------------------------------------------------------------------------
%%  IMPLICATIONS FOR UPCOMING LARGE-SCALE SURVEYS
%%----------------------------------------------------------------------------------------------------------------------------------------------------

With the above upgrades implemented, we expect performance of both the
classification rate and quality to further increase. However, even with our current 
implementation we can cope with upcoming data volumes from large surveys. 
By some estimates, Euclid is expected to obtain
measurable morphology with VIS for approximately $10^6 - 10^7$ galaxies.
Traditional visual classification at the rate achieved today through Galaxy Zoo 
would require approximately 300 years to classify. 
GZX as currently implemented could classify the brightest 20\% of the 
Euclid sample during the six years of its observing mission. 
We predict this is a lower limit as more sophisticated machine learning algorithms
will undoubtedly be capable of further efficiency. 


%%----------------------------------------------------------------------------------------------------------------------------------------------------
%%   CONCLUSIONS 
%%----------------------------------------------------------------------------------------------------------------------------------------------------
\subsection{Conclusions}

We test an innovative system for the efficient classification 
of galaxy morphology tasks that integrates the native ability of the human 
mind to identify the abstract and novel with machine learning algorithms 
that provide speed and brute force.  We demonstrate for the first time that the 
SWAP algorithm, originally developed to identify rare gravitational lenses in the 
Space Warps project, is robust for use in galaxy morphology classification. 
We show that by implementing
SWAP on GZ2 classification data we can increase the rate of classification by a factor
of 4-5, requiring only 90 days of GZ2 project time to classify nearly 80\% of the
entire galaxy sample. 

Furthermore, we have implemented and tested a simple Random Forest algorithm 
and developed a decision engine that delegates tasks between human and 
machine.  We show that even this simple machine is capable of providing significant 
gains in the rate of classification allowing us to retire over 70\% of GZ2 galaxies in 
just 32 days of GZ2 project time, representing an order of magnitude increase in the
classification rate compared to the original GZ2 project. This is achieved without 
sacrificing the quality of classifications as we maintain accuracy well above 90\% 
throughout our simulations. 
Additionally, we have shown that training on a 5-dimensional parameter space of 
traditional non-parametric morphology indicators allows the
machine to identify subjects which humans miss, providing  a complementary 
approach to visual classification. 
The gain in classification speed allows us to tackle the massive amounts of data soon
to be forthcoming from large surveys like LSST, Euclid, and WFIRST. 
%With a few modest upgrades to our GZX framework we are confident this method will be able to handle the gargantuan task of morphology classification of the billions of galaxies soon to be cataloged. 

\bigskip
\bigskip

%%----------------------------------------------------------------------------------------------------------------------------------------------------
%%   ACKNOWLEDGEMENTS
%%----------------------------------------------------------------------------------------------------------------------------------------------------
%\section{Acknowledgements}

MB thanks John Wallin, Steven Bamford, and Boris H{\"a}u{\ss}ler for discussions which helped elucidate things and stuff, and Marc Huertas-Company for several enlightening conversations on machine learning and classification. 
We are grateful to Elisabeth Baeten, Micaela Bagley, Karlen Shahinyan, Vihang Mehta, Karen Masters, Steven Bamford, Kevin Schawinski, and Rebecca Smethurst for providing expert classifications in addition to those provided by the authors. 

MB, CS, LF, KW, and MG gratefully acknowledge support from the US National Science
Foundation Grant AST1413610.  MB acknowledges additional support 
through New College and Oxford University's Balzan Fellowship as well as the University
of Minnesota's Doctoral Dissertation Fellowship. Travel funding was supplied 
to MB, in part, by the University of Minnesota's Thesis Research Travel Grant. CJL recognizes support from a grant from the Science \& Technology Facilities Council (ST/N003179/1). 
BDS acknowledges support from Balliol College, Oxford, and the National Aeronautics and Space Administration (NASA) through Einstein Postdoctoral Fellowship Award Number PF5-160143 issued by the Chandra X-ray Observatory Center, which is operated by the Smithsonian Astrophysical Observatory for and on behalf of NASA under contract NAS8-03060.

%%----------------------------------------------------------------------------------------------------------------------------------------------------
%%   APPENDIX
%%----------------------------------------------------------------------------------------------------------------------------------------------------
\appendix
\label{sec:Appendix}

\section{Measuring Nonparametric Morphological Diagnostics on SDSS Stamps}
\label{sec: measuring morphology}

We obtain $i$-band imaging from SDSS data release 12. Postage stamps are made from the SDSS fields for each galaxy with dimensions of 3 Petrosian radii. Galaxies located within 3 Petrosian radii of the edge of a field were excluded.  Postage stamps undergo a cleaning process whereby nearby sources are identified with SExtractor \citep[ver. 2.8.6;][]{sextractor} and their pixels replaced with values that mimic the background in that region. We compute the following widely adopted nonparametric measurements  of the galaxy light distribution on the cleaned postage stamps:


Concentration is computed as $C = 5\log(r_{80}/ r_{20})$ where \rr{80} and \rr{20} are the radii containing 80\% and 20\% of the galaxy light respectively.  Large values of this ratio tend to indicate disky galaxies, while smaller values correlate with early-type ellipticals. 

Asymmetry quantifies the degree of rotational symmetry in the galaxy light distribution (not necessarily the physical shape of the galaxy as this parameter is not highly sensitive to low surface brightness features). A correction for background noise is applied (as in e.g.~\cite{Conselice2000}), i.e., 
\begin{equation}
A = \frac{\sum_{x,y} |I - I_{180}|}{ 2\sum|I|} - B_{180}
\end{equation}
where $I$ is the galaxy flux in each pixel $(x, y)$, $I_{180}$ is the image rotated by 180 degrees about the galaxy's central pixel, and $B_{180}$ is the average asymmetry of the background. 

The Gini coefficient, $G$,~\citep{Glasser1962, Abraham2003} describes how uniformly distributed a galaxy's flux is.  If $G$ is 0, the flux is distributed homogeneously among all galaxy pixels.; if $G$ is 1,  the light is contained within a single pixel. This term correlates with $C$, however, $G$ does not require that the flux be in the central region of the galaxy.  We follow~\cite{Lotz2004} by first ordering the pixels by increasing flux value, and then computing
\begin{equation}
G = \frac{1}{|\bar X|n(n-1)}\sum_i^n(2i-n-1)|X_i|
\end{equation}
where $n$ is the number of pixels assigned to the galaxy, and $\bar X$ is the mean pixel value. 

\M{20}~\citep{Lotz2004} is the second order moment of the brightest 20\% of the galaxy flux. We compute it as
\begin{eqnarray}
 M_{tot} & = & \sum_i^nf_i[(x_i-x_c)^2 + (y_i-y_c)^2]  \\
 M_{20} & = & \log_{10} (\frac{\sum_iM_i}{M_{tot}}), ~~\textrm{while} \sum_ifi < 0.2f_{tot}
\end{eqnarray}
where M$_{tot}$, the total moment, is computed first and $f_{tot}$ is the total flux. For centrally concentrated objects, \M{20} correlates with $C$ but is also sensitive to bright off-center knots of light. 

Finally, we use the ellipticity, $\epsilon = 1 - b/a$, of the light distribution as measured by SExtractor which computes the semimajor axis $a$ and semiminor axis $b$ from the second-order moments of the galaxy light.  

In total, we measure morphological indicators for 282,350 SDSS galaxies. The relations between these diagnostics for the full sample is shown in Figure~\ref{fig: morph thresh}. The code developed to clean and compute these morphology indicators is open source and can be found at \url{https://github.com/melaniebeck/measure_morphology}.

%%%-------------------------------------------------------
%%%  FIGURE:   Morph Params for 282 K GZ2 subjects
%%%-------------------------------------------------------
\begin{figure*}[t!]

%%   FIGURE:  SWAP -- Retirement Thresholds
%% -------------------------------------------------------------------------------
\includegraphics[width=3.25in]{figures/SWAP_ROC_curve_4paper.png}
\includegraphics[width=3.55in]{figures/morph_params_entire_GZ2_sample.png}
\caption{\textit{Left.} Identifying~\feat~subjects is independent of identifying~\notfeat~subjects.  Both ROC curves use all subjects processed by SWAP where the score used to create the ROC curve is simply each subject's achieved posterior probability. The Featured curve demonstrates how well we identify~\feat~subjects with a threshold of 0.99, while the Not Featured curve demonstrates how well we identify~\notfeat~subjects with a threshold of 0.004. Typically, best performance is achieved with scores that lie closest to the upper left corner.  Our~\feat~threshold is nearly as optimumal as possible though our~\notfeat~threshold could have been slightly better. \textit{Right.} Relation between morphology diagnostics measured for more than 280K SDSS galaxies classified during the GZ2 project. \label{fig: retirement thresholds}}
\label{fig: morph thresh}
%\caption{That's a lot of parameters! \label{fig: machine classified}}
\end{figure*}



%% ----------------------------------------------------------------------------------------------------------------------------------------------
%% DISCUSSION OF CHANGING SWAP PARAMETERS
%% ----------------------------------------------------------------------------------------------------------------------------------------------
\section{Exploring SWAP's Parameter Space}
\label{sec: tweaking swap}

%% -------------------------------------------------------------------------------
%%   FIGURE:  SWAP -- adjust CONFUSION MATRIX
%% -------------------------------------------------------------------------------
\begin{figure*}[t]
\includegraphics[width=3.35in]{figures/GZX_eval_and_retirement_PLPD_spread_4paper_v2.png}
%\caption{GZX/SWAP output as a function of GZ2 project days for a range of initial
%confusion matrix values.  \label{fig: confusionMatrixAnalysis}}
\includegraphics[width=3.35in]{figures/GZX_eval_and_retirement_prior_spread_4paper_v2.png}
\caption{SWAP output does not dramatically change for a range of SWAP parameters.  \textit{Left.} The quality (top) and retirement rate (bottom) when different values of agent confusion matrix are assigned. \textit{Right.} Same as the left panel but for various values of the subject prior probability. Changing the confusion matrix has little impact on the quality of the labels but varies the total number of subjects retired during the run. In contrast, changing the subject prior is more likelyt o affect the quality of the retirement labels rather than the total number retired. \label{fig: tweak swap}}
\end{figure*}

In this Appendix we explore the SWAP parameter space and assess the effects on subject retirement. 


\textbf{Initial agent confusion matrix.} 
In our fiducial simulation each volunteer was assigned an agent with confusion matrix
 (\Pf, \Pn) = $(0.5, 0.5)$, which presumes that volunteers are no better than 
random classifiers.  We perform two simulations wherein we allow (\Pf, \Pn) = $(0.4, 0.4)$, 
slightly obtuse volunteers, and (\Pf, \Pn) = $(0.6, 0.6)$, slightly astute volunteers 
with everything else remaining constant.  
Results of these simulations compared to the fiducial run are shown in 
Figure~\ref{fig: tweak swap}. We find that we are largely insensitive to the 
initial agent confusion matrix probabilities both in terms of the overall number of retired subjects
and in the quality of their SWAP labels. 

Predictably, when (\Pf, \Pn) are low, we retire fewer subjects in the same time frame and 
more subjects when (\Pf, \Pn) are high. This is easy to understand since it takes 
longer for volunteers to become strong, astute classifiers when they are initially 
given values denoting them as obtuse. Even with this handicap, most volunteers 
become astute classifiers by the end of the simulation. Overall,  we retire 
$\sim225$K  $\pm 3.5\%$ subjects as shown by the light blue spread in the bottom
panel of Figure~\ref{fig: tweak swap} where the dashed blue line
denotes the fiducial run. 

The top panel depicts the same quality metrics computed before where the dashed 
lines again denote the fiducial run.  The spread is within a couple per cent for any
metric. Overall we maintain accuracy around $95\%$, as well as completeness of $99\%$
while maintaining purity around $84\%$. This spread can be due to three different
effects: 1) classifying a different subset of subjects, 2) retiring subjects in a different
order, and 3) subjects acquiring a different SWAP label in different simulations. 

We find that SWAP is exceptionally consistent. Of all the subjects retired in
these runs, we find that over 99\% of them are the same subjects between simulations.
Of those consistent between runs, we find that SWAP gives the same label for 
more than 99\% of the subjects. What changes between runs is the order in 
which subjects are classified. In the low (\Pf, \Pn) run, subjects take longer to classify 
compared to the fiducial run (i.e., they retire on a later date in GZ2 project time). 
Subjects in the high (\Pf, \Pn) run retire earlier in GZ2 project time. This can cause
a variation in accuracy, completeness or purity because these values are 
calculated on a day to day basis; if we're working with a slightly different make-up
of subjects on a given day, we can expect to compute different values for these metrics.
These effects each contribute less than one per cent variation and thus we see a 
high level of consistency between these simulations. 



%% -------------------------------------------------------------------------------
%%   FIGURE:  SWAP -- vary SUBJECT PRIOR
%% -------------------------------------------------------------------------------
\begin{figure}[t!]

\end{figure}


\textbf{Subject prior probability,~\p.}
The prior probability assigned to each subject is an educated guess of 
the frequency of that characteristic in the scope of the data at hand. 
For galaxy morphologies, this number should be an estimate of the probability
of observing a desired feature (bar, disk, ring, etc.). In our case, 
we desire to simply find galaxies that are~\feat, however, this is dependent 
on mass, redshift, physical size, etc. The original GZ2 sample was selected
primarily on magnitude and redshift.  As there was no cut on the galaxy size
(with the exception that each galaxy be larger than the SDSS PSF), the sample
includes a large range of galaxy masses and sizes. Thus, designating a single 
prior is not clear-cut. We thus explore how various~\p~affect the SWAP outcome.

We run several simulations where~\p~is allowed to take values 0.2, 0.35, and 0.8 
and compare these to the fiducial run where~\p~= 0.5, everything else remaining constant.
The results are shown in Figure~\ref{fig: priorAnalysis}, where again we find that 
SWAP is consistent in terms of the total number of subjects retired 
during the simulation which varies by only 1\%. 
However, as can be seen in the top panel, the variation in our quality metrics is 
more pronounced and deserves some discussion. 

Firstly, though we are retiring nearly the same number of subjects over the course
of each simulation, they are less consistent than our previous simulations. That is, 
only 95\% of the subjects are common to all runs. Secondly, of those that are 
common, only 94\% receive the same label from SWAP. Changing the prior is more
likely to produce a different label for a given subject than changing the initial 
agent confusion matrix. Finally, there is also a larger spread in the day on which 
a subject is retired when compared to the fiducial run, being nearly equally likely to 
retire `late' or `early' regardless of~\p. These trends all contribute to a broader 
spread in accuracy, completeness, and purity as a function of project time.
We stress, however, that though more substantial than the previous comparison, 
these variations are all within $\pm5\%$. 

We can get a handle on these variations more intuitively by considering the following.
Recall that our retirement thresholds,~\tf~and~\tn, have not changed in these simulations. 
Thus when~\p~is small, the subject probability is already closer to~\tn, and more 
subjects are classified as~\notfeat~compared to the fiducial run.
Similarly, when~\p~is large, some of these same subjects can instead be classified
as~\feat~because the prior probability is already closer to~\tf. Obviously, both 
outcomes cannot be correct and we find that the simulation with~\p~= 0.8 performs
the worst of any run which is a direct reflection of the fact that this prior is not 
suitable for this question nor for this dataset. For the mass, size, and redshift range
of subjects in GZ2, we would not expect that 80\% of them are~\feat. Indeed, 
the best performance is achieved when~\p = 0.35.  This reflects the actual 
distribution of~\feat~subjects in the GZ2 sample as well as being similar to the 
expected proportion of~\feat~galaxies in the local universe, depending on your
definition.
Thus, the take-away here is to choose your prior wisely since a value far from the correct
value can have a significant impact on the quality of your classifications.
 
%% ---------------------------------------------------------------------


\textbf{Retirement thresholds.}
Retirement thresholds are directly related to the time that a subject will spend
in SWAP before retiring.  If we lower~\tf~(and/or raise~\tn), more subjects will be retired
compared to the fiducial run as each subject will have a smaller swath of probability space
in which to bounce back and forth before crossing one of these thresholds.
On the other hand, if we raise~\tf~(and/or lower~\tn), it will take longer for subjects
to cross one of these thresholds. Additionally, this will also increase the likelihood of
some subjects never crossing either threshold as there are always some 
which are nudged back and forth indefinitely through probability space.


What thresholds should one choose? To answer this question, we consider
Figure~\ref{fig: retirement thresholds} which depicts the receiver operating 
characteristic (ROC) curve for our fiducial simulation. The solid line shows the 
curve when considering the~\feat~threshold  while the dotted line
corresponds to~\notfeat. The square and triangle represent our thresholds, 
(\tf,~\tn) = (0.99, 0.004), on that curve at the end of the simulation.  We see that 
~\tf~is nearly optimal but~\tn~could be improved upon. 


Throughout this discussion we have computed quality metrics under the assumption
that the `true' labels provided by GZ2 are accurate. 
This is unlikely to be the case for every subject as~\cite{Willett2013} explicitly caution against using 
the majority volunteer vote fraction as label since some of these are highly uncertain. 
We now turn to a brief discussion of those subjects where SWAP and GZ2 do not agree. 




%% The reference list follows the main body and any appendices.
%% Use LaTeX's thebibliography environment to mark up your reference list.
%% Note \begin{thebibliography} is followed by an empty set of
%% curly braces.  If you forget this, LaTeX will generate the error
%% "Perhaps a missing \item?".
%%
%% thebibliography produces citations in the text using \bibitem-\cite
%% cross-referencing. Each reference is preceded by a
%% \bibitem command that defines in curly braces the KEY that corresponds
%% to the KEY in the \cite commands (see the first section above).
%% Make sure that you provide a unique KEY for every \bibitem or else the
%% paper will not LaTeX. The square brackets should contain
%% the citation text that LaTeX will insert in
%% place of the \cite commands.

%% We have used macros to produce journal name abbreviations.
%% \aastex provides a number of these for the more frequently-cited journals.
%% See the Author Guide for a list of them.

%% Note that the style of the \bibitem labels (in []) is slightly
%% different from previous examples.  The natbib system solves a host
%% of citation expression problems, but it is necessary to clearly
%% delimit the year from the author name used in the citation.
%% See the natbib documentation for more details and options.


\bibliographystyle{apj}
\bibliography{apj-jour,human-machine}

%% This command is needed to show the entire author+affilation list when
%% the collaboration and author truncation commands are used.  It has to
%% go at the end of the manuscript.
%\allauthors

%% Include this line if you are using the \added, \replaced, \deleted
%% commands to see a summary list of all changes at the end of the article.
\listofchanges

\end{document}

%% End of file `sample.tex'.